The partition function $\mathrm{ln}\mathcal{Z}$ can be expressed as a sum over single-particle states. In this case, the logarithm of the grand partition function becomes
\begin{align}
    \ln \mathcal{Z} = \sum_i g_i \ln \left[ 1 + \lambda e^{-\beta \epsilon_i} \right]\,,
\end{align}
where $\epsilon_i$ are the single-particle energy levels, and $g_i$ is the degeneracy of the states. For light quarks, particularly in the limit of vanishing quark masses, the energy levels are proportional to the momentum, $\epsilon_i = |p|$, and the partition function can be rewritten in terms of an integral over momentum space.

The averaged-quark partition function for massless quarks and antiquarks is
\begin{align}
    \mathrm{ln}\mathcal{Z}(T,\lambda_{ud}) = \frac{g_{\mathrm{dof}}V}{6\pi^{2}}T^{3} \left[ \left(1-\frac{2\alpha_{s}}{\pi}\right) \left( \frac{1}{4} \mathrm{ln}^{4}(\lambda_{ud}) + \frac{\pi^{2}}{2} \mathrm{ln}^{2}(\lambda_{ud}) \right) + \left(1-\frac{50\alpha_{s}}{21\pi}\right) \frac{7\pi^{4}}{60} \right]\,.
\end{align}
For details, see Ref.. The quark degrees of freedom are $g_\mathrm{dof}=2\times3\times3$ for spin, color, and flavor multiples.

Test equation:
\begin{align}
    m^2 \left(\mu ^2+\frac{\pi ^2 T^2}{3}\right)+m^3 \left(-\frac{\mu ^2}{2 T}-4 T \log (2)\right)+O\left(m^4\right)
\end{align}

%\AJS{Mass expansion in parameter $\eta/m$ which is different from Elze, 1980. Give it, its own section.}

%\AJS{Because of the CKM matrix, we expect flavor oscillation among quarks which are deconfined in QGP. What is the abundance of each flavor given a coherent scattering length where oscillation is possible during the distance travelled between collisions? Absolute balance is broken becuause of the conversion of lighter-quarks into heavy-quarks. This puts a bias towards heavy quarks as a pure gas of light quarks populates the heavy quark sector.}

%\AJS{To do list: 1. Write down the partition function of the quark species (u,d,ubar,dbar,spinup,spindown). 2. Write down the magnetized microstates. 3. Discuss the chemical potential of the quarks, make a nice plot of density as the chemical potential is very small in this regime.}

%\AJS{Mass corrections to the chemical potential are potentially needed in the regime that we are working in. See. Elze, 1980. "Masses are left for future consideration."}

%\AJS{How to bias the cosmic gas and magnetize it. We need to implement the constraint that local spin is always zero. But as the different particle species have different magnetic movements, the cancellation in magnetization will not be exact.}

%\AJS{Point to Fig. 2 of Fromerth 2012 and mention that the neutrino chemical potential is of a scale with its mass. This is not taken into account and would need further development.}

%%%%%%%%%%%%%%%%%%%%%%%%%%%%%%%%%%%%%%%%
\subsection{Weak field magnetization}
%%%%%%%%%%%%%%%%%%%%%%%%%%%%%%%%%%%%%%%%
Given that \(\rho(s)\) depends on the spin \(s\) as per \req{eq:dimensionless_variables}, with \(\rho(s=\pm1/2) = \rho_{\pm}\), we have
\begin{equation}
\rho_{+}^{2} - \rho_{-}^{2} = -\frac{g |Q| \mathcal{B}}{T^{2}}\,.
\end{equation}
Assuming that the magnetic field \(\mathcal{B}\) is weak, we can expand \(\rho(s)\) around the unperturbed dimensionless mass \(\rho_0\) (the value of \(\rho\) when \(\mathcal{B} = 0\)) as
\begin{equation}
\rho(s) = \rho_0 + s \delta\rho,\qquad \rho_{0} = m/T
\end{equation}
where \( \delta\rho \) is a small perturbation due to the magnetic field. From the relation (keeping only up to first order)
\begin{equation}
\rho_{+}^{2} - \rho_{-}^{2} = ( \rho_0 + \delta\rho )^2 - ( \rho_0 - \delta\rho )^2 = 4 \rho_0 \delta\rho = -\frac{g |Q| \mathcal{B}}{T^{2}},
\end{equation}
we solve for \(\delta\rho\) and obtain
\begin{equation}
\label{eq:weak_field_perturbation}
\delta\rho = -\frac{g |Q| \mathcal{B}}{4 m T}.
\end{equation}
Therefore we can associate \(\delta\rho\) with the standard magnetic dipole energy given by \(|U|\sim\frac{Q}{m}\mathcal{B}\) in thermalized units.

%%%%%%%%%%%%%%%%%%%%%%%%%%%%%%%%%%%%%%%%
\subsubsection{Perturbation in dimensionless mass}
%%%%%%%%%%%%%%%%%%%%%%%%%%%%%%%%%%%%%%%%
We begin by substituting \(\rho(s) = \rho_{0} + s\delta\rho\) into the summation
\begin{equation}
\sum_{s}^{\pm1/2} \left( \epsilon^{2} - \rho(s)^{2} \right)^{3/2} s = \sum_{s}^{\pm1/2} \left( \epsilon^{2} - \left( \rho_{0} + s\delta\rho \right)^{2} \right)^{3/2} s.
\end{equation}
Expanding the squared term inside the parentheses
\begin{equation}
\epsilon^{2} - \left( \rho_{0} + s\delta\rho \right)^{2} = \epsilon^{2} - \rho_{0}^{2} - 2s\rho_{0}\delta\rho - s^{2}\delta\rho^{2}.
\end{equation}
For brevity, we define
\begin{equation}
A \equiv \epsilon^{2} - \rho_{0}^{2}.
\end{equation}
Assuming \(\delta\rho\) is small, we expand the term \(\left( A - 2s\rho_{0}\delta\rho - s^{2}\delta\rho^{2} \right)^{3/2}\) using a Taylor series up to second order in \(\delta\rho\)
\begin{equation}
\left( A - 2s\rho_{0}\delta\rho - s^{2}\delta\rho^{2} \right)^{3/2} \approx A^{3/2} \left[ \frac{3 \delta \rho ^2 \rho _0^2 s^3}{2 A^2}-\frac{3 \delta \rho ^2 s^3}{2 A}-\frac{3 \delta \rho  \rho _0 s^2}{A}+s \right] + \mathcal{O}(\delta\rho^{3}).
\end{equation}
This series expansion is semiconvergent already at second order as evident by the presence of \(\sim A^{-1/2}\) terms that are divergent in the lower bound of the energy integral \(\epsilon\rightarrow\rho\). Therefore, we truncate after first order in our expansion.

%%%%%%%%%%%%%%%%%%%%%%%%%%%%%%%%%%%%%%%%
\subsubsection{Summation over spin states}
%%%%%%%%%%%%%%%%%%%%%%%%%%%%%%%%%%%%%%%%
We now perform the summation on \(s = \pm1/2\). Note that due to the symmetry of the spin states, certain terms will cancel out:
\begin{equation}
\sum_{s}^{\pm1/2} \left( \epsilon^{2} - \rho(s)^{2} \right)^{3/2} s \approx A^{3/2} \sum_{s}^{\pm1/2} \left[ -\frac{3 s^3}{2 A}\delta \rho ^2 - \frac{3 \rho _0 s^2}{A}\delta \rho + s \right] + \mathcal{O}(\delta\rho^{2}).
\end{equation}
Evaluating the sum term-by-term, we find
\begin{align}
(\delta\rho)^{0}&:\sum_{s}^{\pm1/2} s = \frac{1}{2} - \frac{1}{2} = 0,\\
(\delta\rho)^{1}&:\sum_{s}^{\pm1/2} - \frac{3 \rho _0 s^2}{A}\delta \rho = -\frac{3 \rho _0}{2 A}\delta \rho.
\end{align}
Combining these results, the summation up to first order in \(\delta\rho\) is
\begin{equation}
\label{eq:spin_summation_perturbation}
\sum_{s}^{\pm1/2} \left( \epsilon^{2} - \rho(s)^{2} \right)^{3/2} s \approx -\frac{3\delta \rho}{2} \rho _0 \sqrt{\epsilon ^2-\rho _0^2} + \mathcal{O}(\delta\rho^{2}).
\end{equation}
Higher-order terms in \(\delta\rho\) introduce corrections that become progressively smaller for \(\delta\rho \ll 1\), justifying the truncation of the Taylor series at first order for small perturbations. This approximation facilitates analytical progress in studying the behavior of the system under small deviations from the unperturbed mass \(\rho_{0}\).

For a small perturbation \(\rho\approx\rho_{0}\) at lowest order the integration bounds are unchanged
\begin{equation}
    \sum_{s}^{\pm1/2} \int^{\infty}_{\rho(s)} \ldots d\epsilon = \int^{\infty}_{\rho_{+}} \ldots d\epsilon + \int^{\infty}_{\rho_{-}} \ldots d\epsilon \approx 2\int^{\infty}_{\rho_{0}} \ldots d\epsilon
\end{equation}
Using \req{eq:weak_field_perturbation}, \req{eq:spin_summation_perturbation} and \req{eq:mag_form}, the weak field relativistic magnetization is
\begin{align}
    \label{eq:mag_form_weak}
    \lim_{\delta\rho\ll\rho_{0}}\mathcal{M} &\approx -\frac{g^2 Q^2 \mathcal{B}}{16 T^3}\cdot\frac{ N_{\mathrm{dof}} V T^{3}}{2\pi^{2}} \sum_{\sigma}^{\pm1} \int_{\rho}^{\infty} d\epsilon \, \sqrt{\epsilon ^2-\rho _0^2} \\*
    &\times\left[ \frac{F(\epsilon - \sigma\xi)\left(1 - F(\epsilon - \sigma\xi)\right)}{\epsilon} + \frac{F(\epsilon - \sigma\xi)}{\epsilon^{2}} \right].
\end{align}
